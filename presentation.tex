\documentclass{beamer}

\setbeamertemplate{section in toc}[sections numbered]
\setbeamertemplate{subsection in toc}[subsections numbered]

% american mathematical society
\usepackage{amsmath,amsthm,amssymb}

% reversed cases enviroment
\newenvironment{rcases}{\left.\begin{aligned}}{\end{aligned}\right\rbrace}

% drawing functionality
\usepackage{tikz}
\usetikzlibrary{matrix,patterns}

% equation numbering
\numberwithin{equation}{section}

% theorem types
\newtheorem{proposition}{Proposition}

% math operators
\DeclareMathOperator*{\argmin}{argmin}
\DeclareMathOperator*{\var}{Var}
\DeclareMathOperator*{\cov}{Cov}
\DeclareMathOperator{\VaR}{VaR}
\DeclareMathOperator{\cvar}{CVaR}
\DeclareMathOperator{\supp}{supp}
\DeclareMathOperator{\dist}{dist}

% additional mathematical fonts
\usepackage{mathrsfs}

% remove navigation bar
\beamertemplatenavigationsymbolsempty

% add slide numbers
\setbeamertemplate{footline}[frame number]

% title
\title{Measure Theory}
\subtitle{Exam}
\author{Kasper Rosenkrands}
\institute{Aalborg University}
\date{E20}

% toc at each section
\AtBeginSection[]
{
  \begin{frame}
    \frametitle{Table of Contents}
    \tableofcontents[currentsection, hideothersubsections]
  \end{frame}
}

% definition of comment function
\newcommand{\comment}[1]{
    \begin{center}
        \colorbox{yellow}{
            \textsf{
                \textbf{#1}
            }
        }
    \end{center}
}
\newcommand{\task}[1]{
    \begin{center}
        \colorbox{red}{
            \textsf{
                \textbf{#1}
            }
        }
    \end{center}
}

\newenvironment{frame2}{\begin{frame}\frametitle{{\normalsize \secname} \\ {\large \subsecname}}}{\end{frame}}

\begin{document}

\frame{\titlepage}

\begin{frame}
\frametitle{Table of Contents}
\tableofcontents[hideallsubsections]
\end{frame}

\section{Conditional Expectation}

\section{Brownian Motions}

\section{Martingales \& Quadratic Variation}

\subsection{Prerequisties}

\begin{frame2}
    \task{Define these things}
    \begin{itemize}
        \item Brownian Motions (only briefly as this is topic 2)
        \item Martingale
        \item Natural Filtration
    \end{itemize}
\end{frame2}

\begin{frame2}
    Let $(M_t)_{t\in I}$ be an adapted $\mathbb{R}$-valued stochastic process such that $M_t: \ \Omega \rightarrow \mathbb{R}$ is $\mathbb{P}$-integrable for every $t\in I$.
    \begin{itemize}
        \item<-1> $M$ is called a $(\mathcal{F}_t)_{t\in I}$-martingale iff
        \begin{align}
            \mathbb{E}^{\mathcal{F}_s}[M_t] = M_s, \quad \mathbb{P}-\text{a.s.\! for all } s,t \in I \text{ with } s\leq t.
        \end{align}
        \item<2-> $M$ is called a $(\mathcal{F}_t)_{t\in I}$-submartingale iff
        \begin{align}
            \mathbb{E}^{\mathcal{F}_s}[M_t] \leq M_s, \quad \mathbb{P}-\text{a.s.\! for all } s,t \in I \text{ with } s\leq t.
        \end{align}
        \item<3-> Analogously for supermartingale.
    \end{itemize}
\end{frame2}

\begin{frame2}
    
\end{frame2}

\subsection{One-dimensional Brownian Motions}

\begin{frame2}
    If $B$ is a one-dimensional $(\mathcal{F}_t)_{t\geq 0}$-Brownian motion, then $B$ is a $(\mathcal{F}_t)_{t\geq 0}$-martingale.

    In particular, every one-dimensional standard Brownian motion $B$ is a martingale with respect to its natural filtration $(\mathcal{B}_t^B)_{t\geq 0}$.
    \vspace{5pt}
    \hrule
    \vspace{5pt}
    %\task{Repeat Remark 6.7}
    Remark 6.7. Let $(M_t)_{t \in I}$ be a $(\mathcal{F}_t)_{t\in I}$-martingale and denote by $(\Omega, \tilde{\mathcal{F}},(\tilde{\mathcal{F}}_t)_{t\in I},\tilde{\mathbb{P}})$ the completion of $(\Omega, \mathcal{F},(\mathcal{F}_t)_{t\in I},\mathbb{P})$.
    Then $(M_t)_{t\in I}$ is a $(\tilde{\mathcal{F}}_t)_{t\in I}$-martingale as well, provided that $(\Omega, \tilde{\mathcal{F}},(\tilde{\mathcal{F}}_t)_{t\in I},\tilde{\mathbb{P}})$ is chosen as the underlying filtered probability space.
    \vspace{5pt}
    \hrule
    \vspace{5pt}
    By virtue of Remark 6.7 we can further conclude that every one-dimensional standard Brownian motion $B$ is a martingale on the standard filtered probability space $(\Omega, \tilde{\mathcal{F}},(\tilde{\mathcal{F}_t^B})_{t\geq 0},\tilde{\mathbb{P}})$ obtained by completing $(\Omega, \mathcal{F},(\mathcal{F}_t^B)_{t\geq 0},\mathbb{P})$.
\end{frame2}

\begin{frame2}
    Let us first recall that
    \begin{align}
        \int_{\Omega} |B_t| \, d\mathbb{P} &= \frac{1}{(2\pi t)^{1/2}}\int_\mathbb{R} |x| e^{-x^2/2t} \, dx < \infty, \\
        \mathbb{E}[B_t] &= \frac{1}{(2\pi t)^{1/2}}\int_\mathbb{R}xe^{-x^2/2t} \, dx,
    \end{align}
    for every $t > 0$.
\end{frame2}

\section{Itôs Formula}

\section{Stochastic Integrals}

\section{Girsanov-Transformation}

\end{document}
