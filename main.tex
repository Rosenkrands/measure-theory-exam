\documentclass{article}
\usepackage{amsmath,amsthm,amssymb}
\usepackage{mathabx}
\newtheorem{theorem}{Theorem}[section]
\newtheorem{definition}{Definition}[section]
\numberwithin{equation}{section}

\begin{document}

\thispagestyle{empty}
\begin{center}
{\huge\textbf{Measure Theory}}\\[2mm]
{\Large Kasper Rosenkrands}\\[2cm]
{\large MATØK7}\\[2mm]
{\large Fall 2020}
\end{center}

\newpage

\tableofcontents

\newpage

\pagenumbering{arabic}

\section*{Notation}
\subsection*{Measure with densities}
\begin{theorem}[Measures with densities]
    Let $(\Omega, \mathcal{A}, \mu)$ be a measure space and let $f : \Omega \rightarrow [0,\infty]$ be measureable.
    Then
    \begin{align*}
        (f \odot \mu)(A) := \int_A f \, d\mu = \int_\Omega \chi_A f \, d\mu, \quad A \in \mathcal{A},
    \end{align*}
    defines a new measure $f \odot \mu: \mathcal(A) \rightarrow [0,\infty]$, called the measure with density $f$ with respect to $\mu$.
    For every $N\in\mathcal{A}$, the following implication holds,
    \begin{align*}
        \mu(N) = 0 \ \Rightarrow \ (f\odot\mu)(N) = 0.
    \end{align*}
\end{theorem}

\subsection*{Product $\boldsymbol{\sigma}$-algrebras}
\begin{definition}
    Let $n\in\mathbb{N}$, $n \geq 2$, and suppose that, for every $i \in \{1,\ldots,n\}$, we are given a measurable space $(\Omega_i, \mathcal{A}_i)$.
    \begin{itemize}
        \item The smallest $\sigma$-algebra on $\bigtimes_{i=1}^n \Omega_i$ containing
        \begin{align*}
            \mathcal{A}_1 \ast \cdots \ast \mathcal{A}_n := 
            %
            \left\{A_1 \times \cdots \times A_n \, | \, A_1 \in \mathcal{A}_1, \ldots, A_n \in \mathcal{A}_n\right\} %
            \subset \mathcal{P}\left(\bigtimes_{i=1}^n \Omega_i\right)
        \end{align*}
        is called the product $\sigma$-algebra defined by measns of $\mathcal{A}_1,\ldots,\mathcal{A}_n$.
        It is denoted by
        \begin{align*}
            \bigotimes_{i=1}^n \mathcal{A}_i := \sigma\left(\mathcal{A}_1 \ast \cdots \ast \mathcal{A}_n\right).
        \end{align*}
        \item Let $\Gamma$ be a set and let $f_i \, : \, \Gamma \rightarrow \Omega_i$ be an arbitrary map for every $i \in \{1, \ldots, n\}$.
        Then the smallest $\sigma$-algebra on $\Gamma$ turning all maps $f_1,\ldots,f_n$ into measurable maps, i.e.,
        \begin{align*}
            \sigma(f_1, \ldots, f_n) := \sigma\left(\left\{f_i^{-1}(A_i)\, | \, A_i \in \mathcal{A}_i, i \in {1,\ldots,n}\right\}\right),
        \end{align*}
        is called the initial $\sigma$-algebra generated by $f_1, \ldots,f_n$.
    \end{itemize}
\end{definition}

\newpage

\section{Betingede forventningsværdier}

\newpage

\section{Processer med uafhængige of stationære tilvækst, specielt standard brownske bevægelser}

\newpage

\section{Martingaler og kvadratisk variation}

\subsection*{relevante dele til forelæsning 9}
\begin{itemize}
    \item afsnit 5.8.1
    \begin{itemize}
        \item def 5.101
        \item eks 5.103
        \item def 5.104
        \item eks 5.105
        \item sæt 5.110
    \end{itemize}
    \item afsnit 5.8.2
    \begin{itemize}
        \item def 5.112
        \item bem 5.114
        \item bem 5.115
        \item sæt 5.118
        \item sæt 5.120
    \end{itemize}
    \item afsnit 5.8.3
    \begin{itemize}
        \item def 5.122
        \item bem 5.123
        \item sæt 5.125
        \item sæt (med def) 5.126
    \end{itemize}
\end{itemize}

\subsection{Martingales}

\subsubsection{One-dimensional Brownian Motion}
Let $B$ be a one-dimensional $(\mathcal{F}_t)_{t\geq 0}$-Brownian motion.
Then $B$ is a $(\mathcal{F}_t)_{t\geq 0}$ martingale.
In particular, every one-dimensional standard Brownian motion $B$ is a martingale with respect to its natural filtration $(\mathcal{B}_t^B)_{t\geq 0}$.
By virtue of Remark 6.7 we can further conclude that every one-dimensional standard Brownian motion $B$ is a martingale on the standard filtered probability space $(\Omega, \tilde{\mathcal{F}},(\tilde{\mathcal{F}_t^B})_{t\geq 0},\tilde{\mathbb{P}})$ obtained by completing $(\Omega, \mathcal{F},(\mathcal{F}_t^B)_{t\geq 0},\mathbb{P})$.

Let us first recall that
\begin{align}
    \int_{\Omega} |B_t| \, d\mathbb{P} = \frac{1}{(2\pi t)^{1/2}}\int_\mathbb{R} |x| e^{-x^2/2t} \, dx < \infty,
    \mathbb{E}[B_t] = \frac{1}{(2\pi t)^{1/2}}\int_\mathbb{R}xe^{-x^2/2t} \, dx,
\end{align}
for every $t > 0$.
Since $B_0 = 0$, $\mathbb{P}-a.s.$, it follows in particular that $B_t$ is $\mathbb{P}$-integrable with $\mathbb{E}[B_t] = 0$ for all $t \geq 0$.
For all $0 \leq s \leq t < \infty$, we further observe that, $\mathbb{P}-a.s.$,
\begin{align}
    \mathbb{E}^{\mathcal{F}_s}[B_t] = \mathbb{E}^{\mathcal{F}_s}[B_t - B_s] + \mathbb{E}^{\mathcal{F}_s} = \mathbb{E}[B_t - B_s] + B_s = \mathbb{E}[B_t] - \mathbb{E}[B_s] + B_s, 
\end{align}
since the increment $B_t - B_s$ is $\mathcal{F}_s$-independent and $B_s$ is $\mathcal{F}_s$-measurable.

\newpage

\section{Itô-formlen}

\newpage

\section{Girsanov-transformationen}

\end{document}